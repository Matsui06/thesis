\begin{center}
 {\large \textrm{概要}}
\end{center}

%軽い原子核の束縛エネルギーや少数核子系散乱における観測量等、様々な物理量の記述には三つの核子間で同時に相互作用が起こる三体核力の効果が必要不可欠であることが知られている。我々はこの三体核力の発現性について詳細に調べるために、偏極$^3$He標的を用いた$70~{\rm MeV}$の陽子--$^3$He散乱実験による$^3$He偏極分解能$A_y$の測定を計画している。$A_y$を測定するためには偏極$^3$He標的の偏極度の評価が不可欠であるが、現在我々が$^3$He偏極度の測定方法として採用している高速断熱通過-核磁気共鳴(AFP-NMR)法のみでは、高精度で$^3$He偏極度を得ることができない。そこで、本研究ではRbの電子スピン共鳴(ESR)を利用した新たな$^3$He偏極度測定システムの開発を行った。\\
% $^3$He原子核を偏極させる方法としては、スピン交換光ポンピング法を採用している。これは円偏光レーザーによって静磁場中のRb原子を偏極させ、偏極したRb原子と$^3$He原子核がスピン交換反応をすることで$^3$He原子核を偏極させる方法である。この時、混合気体中のRbのESR周波数が偏極した$^3$He原子核によってシフトすることが知られており、この周波数シフトを測定することで$^3$He偏極度を求めることが出来る。本研究では、このESR周波数シフト測定システムの開発を行った。\\
% 開発した測定システムにより、ESR周波数シフトを確認することに成功した。確認された周波数シフトから、$^3$He偏極度の絶対値を求めた。また同時にAFP-NMR法による$^3$He偏極度測定も行い、AFP-NMR法で得られるNMR信号強度の$^3$He偏極度に対する較正を行った。\\
% 東北大学CYRICにおいて、$70~{\rm MeV}$の陽子--$^3$He弾性散乱実験を行った。本研究で開発したシステムによる較正結果から、実験中における$^3$He偏極度は$10〜11$%程度であった。また散乱陽子の検出は、ビーム方向に対して左右それぞれ、実験室系での角度$55^\circ$および$70^\circ$に設置した検出器によって行った。測定結果および実験中に得られた$^3$He偏極度の絶対値から、$^3$He偏極分解能$A_y$の値を算出した。

